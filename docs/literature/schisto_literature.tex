\documentclass[10,portrait]{article}
\usepackage{lmodern}
\usepackage{amssymb,amsmath}
\usepackage{ifxetex,ifluatex}
\usepackage{fixltx2e} % provides \textsubscript
\ifnum 0\ifxetex 1\fi\ifluatex 1\fi=0 % if pdftex
  \usepackage[T1]{fontenc}
  \usepackage[utf8]{inputenc}
\else % if luatex or xelatex
  \ifxetex
    \usepackage{mathspec}
  \else
    \usepackage{fontspec}
  \fi
  \defaultfontfeatures{Ligatures=TeX,Scale=MatchLowercase}
\fi
% use upquote if available, for straight quotes in verbatim environments
\IfFileExists{upquote.sty}{\usepackage{upquote}}{}
% use microtype if available
\IfFileExists{microtype.sty}{%
\usepackage[]{microtype}
\UseMicrotypeSet[protrusion]{basicmath} % disable protrusion for tt fonts
}{}
\PassOptionsToPackage{hyphens}{url} % url is loaded by hyperref
\usepackage[unicode=true]{hyperref}
\PassOptionsToPackage{usenames,dvipsnames}{color} % color is loaded by hyperref
\hypersetup{
            pdftitle={Key literature on Schistosoma and host-parasite systems},
            colorlinks=true,
            linkcolor=blue,
            citecolor=red,
            urlcolor=blue,
            breaklinks=true}
\urlstyle{same}  % don't use monospace font for urls
\usepackage[margin=1in]{geometry}
\usepackage[]{biblatex}
\IfFileExists{parskip.sty}{%
\usepackage{parskip}
}{% else
\setlength{\parindent}{0pt}
\setlength{\parskip}{6pt plus 2pt minus 1pt}
}
\setlength{\emergencystretch}{3em}  % prevent overfull lines
\providecommand{\tightlist}{%
  \setlength{\itemsep}{0pt}\setlength{\parskip}{0pt}}
\setcounter{secnumdepth}{0}
% Redefines (sub)paragraphs to behave more like sections
\ifx\paragraph\undefined\else
\let\oldparagraph\paragraph
\renewcommand{\paragraph}[1]{\oldparagraph{#1}\mbox{}}
\fi
\ifx\subparagraph\undefined\else
\let\oldsubparagraph\subparagraph
\renewcommand{\subparagraph}[1]{\oldsubparagraph{#1}\mbox{}}
\fi

% set default figure placement to htbp
\makeatletter
\def\fps@figure{htbp}
\makeatother


\title{Key literature on \emph{Schistosoma} and host-parasite systems}
\author{Matthew
Malishev\textsuperscript{1}*\\[2\baselineskip]\emph{\textsuperscript{1}
Department of Biology, Emory University, 1510 Clifton Road NE, Atlanta,
GA, USA, 30322}}
\date{}

\begin{document}
\maketitle

{
\hypersetup{linkcolor=black}
\setcounter{tocdepth}{3}
\tableofcontents
}
\newpage   

Date: 2019-02-22\\
R version: 3.5.0\\
*Corresponding author:
\href{mailto:matthew.malishev@emory.edu}{\nolinkurl{matthew.malishev@emory.edu}}\\
This document can be found at \url{https://github.com/darwinanddavis/}

\newpage  

\subsection{Overview}\label{overview}

Key literature and areas of research in \emph{Schistosoma} and
host-parasite systems.

\subsection{Notes}\label{notes}

\textbf{Models with schisto}

Aim: To account for time lag in population cycles based on things like
resources and size structure

\begin{itemize}
\tightlist
\item
  Charles King\\
\item
  Mark EJ Woolhouse (1991, 1992)\\
\item
  David Rollinson
\end{itemize}

\newpage  

\begin{center}\rule{0.5\linewidth}{\linethickness}\end{center}

\subsection{Habitat and resources}\label{habitat-and-resources}

\href{Southgate\%20VR\%20(1997)\%20Schistosomiasis\%20in\%20the\%20Senegal\%20River\%20Basin:\%20Before\%20and\%20after\%20the\%20construction\%20of\%20the\%20dams\%20at\%20Diama,\%20Senegal\%20and\%20Manantali,\%20Mali\%20and\%20future\%20pros-\%20pects.\%20J\%20Helminthol\%2071(2):125–132}{Southgate
VR (1997) Schistosomiasis in the Senegal River Basin: Before and after
the construction of the dams at Diama, Senegal and Manantali, Mali and
future pros- pects. J Helminthol 71(2):125--132.}

\begin{quote}
From Sokolow et al 2015 PNAS
\end{quote}

\newpage  

\begin{center}\rule{0.5\linewidth}{\linethickness}\end{center}

\subsection{Host population}\label{host-population}

Rumi and Hamann 1992 - size structure dynamics of Biomphalaria

Loreau 1987 - Size structure in natural Biomphalaria population

Ituarte 1989 - Growth dynamics of B glabrata in the field

\newpage    

\begin{center}\rule{0.5\linewidth}{\linethickness}\end{center}

\subsection{Cerc stuff}\label{cerc-stuff}

\subsubsection{Cerc host choice}\label{cerc-host-choice}

Snail-host-finding by Miracidia and Cercariae: Chemical Host Cues

Langeloh (2018) Relative importance of chemical attractiveness to
parasites for susceptibility to trematode infection

Seppala (2015) Quality attracts parasites: host condition-dependent
chemo-orientation of trematode larvae

Sukhdeo (2004) Trematode behaviours and the perceptual worlds of
parasites

\newpage  

\begin{center}\rule{0.5\linewidth}{\linethickness}\end{center}

\subsection{Human health stats}\label{human-health-stats}

\href{}{Steinmann P, Keiser J, Bos R, Tanner M, Utzinger J (2006)
Schistosomiasis and water resources development: Systematic review,
meta-analysis, and estimates of people at risk. Lancet Infect Dis
6(7):411--425}

\begin{quote}
From Sokolow et al 2015 PNAS\\
Schistosomiasis infects an estimated 220--240 million people globally,
and 790 million are at risk for infection, more than 90\% of whom are in
Sub-Saharan Africa (14).
\end{quote}

\href{www.who.int/neglected_diseases/preventive_chemotherapy/databank/en/}{WHO
(2015) Preventive Chemotherapy and Transmission Control (PCT)
databank}\\
\href{}{World Health Assembly (2012) Elimination of Schistosomiasis in
WHA65/2012/REC/1 Sixty-Fifth World Health Assembly: Resolutions and
Decisions Annexes (WHO, Geneva)}

\href{}{Cheever AW, Macedonia JG, Mosimann JE, Cheever EA (1994)
Kinetics of egg pro- duction and egg excretion by Schistosoma mansoni
and S. japonicum in mice infected with a single pair of worms. Am J Trop
Med Hyg 50(3):281--295}

\begin{quote}
From Sokolow et al 2015 PNAS\\
Each infected snail sheds thousands of cer- cariae, which seek and
penetrate human skin. After entering the skin, the parasites migrate to
the blood vessels of the intestines (S. mansoni) or urinary bladder (S.
hematobium), where female worms lay 350--2,200 eggs per day (15)
\end{quote}

\href{}{Jobin WR, Negrón-Aponte H, Michelson EH (1976) Schistosomiasis
in the Gorgol Valley of Mauritania. Am J Trop Med Hyg 25(4):587--594}

\begin{quote}
From Sokolow et al 2015 PNAS
\end{quote}

\begin{quote}
Death from liver failure or bladder cancer can be preceded by chronic
anemia, cognitive im- pairment in children, growth stunting,
infertility, and a higher risk of contracting HIV in women (17, 18).
\end{quote}

\href{https://www.biorxiv.org/content/early/2018/08/11/387969}{Whole
genome sequencing and morphological analysis of the human-infecting
schistosome emerging in Europe reveals a complex admixture between
Schistosoma haematobium and Schistosoma bovis parasites.}

\href{}{Hotez PJ, Fenwick A, Kjetland EF (2009) Africa's 32 cents
solution for HIV/AIDS. PLoS Negl Trop Dis 3(5):e430}

\begin{quote}
Notes
\end{quote}

\href{www.who.int/neglected_diseases/preventive_chemotherapy/databank/}{WHO
(2015) Preventive Chemotherapy and Transmission Control (PCT) databank.}

\begin{quote}
Notes
\end{quote}

\href{}{WHO (2011) Schistosomiasis: Progress Report 2001-2011 and
Strategic Plan 2012-2020 (WHO, Geneva}.

\begin{quote}
Notes
\end{quote}

\href{}{Bockarie MJ, Kelly-Hope LA, Rebollo M, Molyneux DH (2013)
Preventive chemother- apy as a strategy for elimination of neglected
tropical parasitic diseases: Endgame challenges. Philos Trans R Soc Lond
B Biol Sci 368(1623):20120144.}

\begin{quote}
Notes
\end{quote}

\href{}{Gray DJ, et al. (2010) Schistosomiasis elimination: Lessons from
the past guide the future. Lancet Infect Dis 10(10):733--736}

\begin{quote}
Notes
\end{quote}

\href{}{Fenwick A, Savioli L (2011) Schistosomiasis elimination. Lancet
Infect Dis 11(5):346, author reply 346--347}

\begin{quote}
Notes
\end{quote}

\href{}{Zhang Z, Jiang Q (2011) Schistosomiasis elimination. Lancet
Infect Dis 11(5):345, au- thor reply 346--347}

\begin{quote}
Notes
\end{quote}

\href{}{World Health Assembly (2012) Elimination of Schistosomiasis in
WHA65/2012/REC/1 Sixty-Fifth World Health Assembly: Resolutions and
Decisions Annexes (WHO, Geneva).}

\begin{quote}
Notes
\end{quote}

\newpage  

\begin{center}\rule{0.5\linewidth}{\linethickness}\end{center}

\subsection{Starvation}\label{starvation}

\href{https://www.ncbi.nlm.nih.gov/pmc/articles/PMC4945359/}{Nelson et
al. 2016 Effects of abnormal temperature and starvation on the internal
defense system of the schistosome-transmitting snail Biomphalaria
glabrata}

\begin{quote}
Notes
\end{quote}

\href{https://besjournals.onlinelibrary.wiley.com/doi/pdf/10.1111/1365-2656.12195}{Andre´
Gergs and Tjalling Jager 2014 Body size-mediated starvation resistance
in an insect predator}

\begin{quote}
Notes
\end{quote}

\newpage  

\begin{center}\rule{0.5\linewidth}{\linethickness}\end{center}

\subsection{Molluscicide}\label{molluscicide}

\href{}{Sokolow et al 2018 To Reduce the Global Burden of Human
Schistosomiasis, Use `Old Fashioned' Snail Control}

\begin{quote}
Contains table of papers looking at schisto control programs and
strategies
\end{quote}

\newpage  

\begin{center}\rule{0.5\linewidth}{\linethickness}\end{center}

\subsection{Immunity}\label{immunity}

\href{https://onlinelibrary.wiley.com/doi/pdf/10.1111/ele.12229}{Cressler
et al 2014 Disentangling the interaction among host resources, the
immune system and pathogens}

\begin{quote}
Notes
\end{quote}

\newpage  

\begin{center}\rule{0.5\linewidth}{\linethickness}\end{center}

\subsection{Predation}\label{predation}

\href{http://www.pnas.org/content/pnas/early/2015/07/15/1502651112.full.pdf}{Sokolow
et al 2015 Reduced transmission of human schistosomiasis after
restoration of a native river prawn that preys on the snail intermediate
host}

\begin{quote}
`Reinfection after treatment is a problem that plagues efforts to
control parasites with complex transmission pathways, such as
schistosomiasis, which affects at least 220 million people worldwide and
requires an obligate snail intermediate host.'
\end{quote}

\begin{quote}
\end{quote}

\newpage  

\subsection{Detritus and algae}\label{detritus-and-algae}

K. R. Reddy \& W. F. DeBusk. 1991. Decomposition of water hyacinth
detritus in eutrophic lake water. Hydrobiologib 211: 101-109.

\begin{quote}
Detritus production rates
\end{quote}

K. K. Moorhead, K. R. Reddy \& D. A. Graetz 1988 Water hyacinth
productivity and detritus accumulation. Hydrobiologia 157: 179-185.

\begin{quote}
Detritus production rates
\end{quote}

\printbibliography

\end{document}
